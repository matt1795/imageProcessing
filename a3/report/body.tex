\begin{enumerate}[font=\bfseries]

    \item \textbf{For the image below (let’s call it image A), apply a 3x3
    averaging low pass filter and call it image B. Apply the same averaging
    filter to Image B to produce image C.  Draw image B and C. Explain what
    happens when you repeatedly apply an averaging filter to an image?}

    % Make script to generate image and apply filter.

    \begin{figure}[H]
	\centering
	\includegraphics[scale=0.75]{q1.png}
	\caption{Consecutive 3x3 Average Filter}
    \end{figure}

    \item \textbf{In Image shown here, each bar has a width of 6 pixels. The
    gaps between bars are 19 pixels. Explain the result of filtering this image
    using an averaging mask of 25x25 and 20x20}

    % Make script to generate image and apply filter.

    \begin{figure}[H]
	\centering
	\includegraphics[scale=0.6]{q2.png}
	\caption{25x25 vs. 20x20 Average Filter}
    \end{figure}

    \item \textbf{In spatial filtering, a mask is applied to top-left corner of
    an image and then the center of the mask will be moved through the image. At
    each location, the sum of product of the mask coefficients with the
    corresponding pixel values at that location is calculated. Then the pixel
    value of the image corresponding with the center of the mask will be
    updated. Considering an averaging mask of size $n\times n$with coefficients
    of $\frac{1}{n^2}$.  Calculate the minimum computational complexity of
    applying the averaging filter to an image.  Remember that for an averaging
    mask all the coefficients are one considering a scale factor of
    $\frac{1}{n^2}$.}


    \item \textbf{A digital image $f(x,y)$ is  changed to a  black and white
    image $b(x,y)$.  $b(x,y)$ is passed through a 3x3 blurring filter as shown
    here to produce output image $o(x,y)$. What are possible intensity values in
    the output?}

    It important to note that the sum of the coefficients is one. In the case
    where all the pixel values are 255, the output would b 255. When all the
    values are 0, the output is zero. This can be mapped to all values, and so
    the possible values of the output are anywhere from 0 - 255.

\end{enumerate}

\pagebreak

\appendix

\section{Code Listings}

\subsection{Script for Question 1}

\lstinputlisting[style=mattLab]{../q1.m}

\pagebreak

\subsection{Script for Question 2}

\lstinputlisting[style=mattLab]{../q2.m}
