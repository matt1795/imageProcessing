\section{Summary}

In this lab we investigate DCT transforms. This transform is similar to the DFT
because it takes a function in the spatial domain and transforms it into the
frequency domain. It is often used in lossy compression schemes such as mp3 and
jpeg for audio and image information respectively. DCT is used in these lossy
compression schemes because it has been found that cosines are better for
compacting data.

An interesting characteristic of the DCT (likely due to it only having real
valued coefficients) is that it has the DC component in one corner and the high
frequency components expand from there. In DFT the DC component was located at
each corner. I would be interested in looking at filter design using DCT, it
would not likely be too difficult, but the combination of a lossy conversion
with a filter seems interesting.
