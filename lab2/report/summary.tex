\section{Summary}

In this lab we covered several topics pertaining to point processing. Point
processing is a method of image processing where pixels are translated, read,
and altered.

In one of our first exercizes we add a constant offset of 60 to an image. This
increases the brightness of the image noticeably and it distorts it as well.
While note completely obvious, we are losing information because some pixel
values are, what we like to say, "hitting the rails" at the value 255. This is
the maximum brightness and we cannot go beyond it. If we were given this value
by itself, we would not be able to determine its original value precisely, and
so information was lost.

In another exercize we use a power law, or gamma transformation. I have
previously read that gamma correction is an image acquisition technique that is
used to correct brightness levels in an image taken by a camera to better match
the image when the scene is seen by human eyes.

Continuing with our exercize... The gamma transformation seems to essentially
change brightness of the image. It is definitely not like a constant offset or
linear expansion, it is able to do this without changing the bounds of the image
and in doing so, loses little information.

Logarithmic transformations are used when a large range of values exist, if we
had a large range of values, say 16-bit, and most values were below 1000, then
we could use a logarithmic transform to better see details throughout the
brightness range. I can definitely see application for this type of transform in
radiology, like when NASA publishes images of gas clouds. In the exercize we are
using this transform on the tire image. At one point we are able to see radial
lines on the tire. This happened because the difference between the original
brightness of the lines vs. the background was very small. The logarithmic
transform "zoomed in" on this range and increased the difference between the two
enough that we were able to notice. It was not noticed earlier because the
previous transforms simply did not "zoom in" on this range of value enough.

We then have a piecewise function that "negates" a value if said value is below
a threshold. The act of "negating" does have the effect of mirroring the value
onto the other side of the possible value range. This is brought on by the fact
that we are using unsigned 8-bit numbers and two's complement when using
negative numbers. Let's say we have the value 1, which is very dark. If we turn
it into -1, the 8-bit two's complement value is 11111111, the unsigned value of
that is 255, the brightest value. Interesting this piecewise function will never
alter a pixe with value 0.

In our example, a threshold value of 0 does nothing, 255 gives us a negative of
the image, and 100 only partially negates the image, making those parts much
brighter.
