\begin{enumerate}[font=\bfseries]

    \item \textbf{An image has the PDF $p_r(r)=Ae^{-r}$. It is desired to
    transform the intensity values of this image so that its PDF changes to
    $p_z(z)=Bze^{-z^{2}}$. Assume continuous quantities and find the
    transformation to perform this task.}
    
    \begin{align*}
	p_r(r) &= Ae^{-r} \\
	p_z(z) &= Bze^{-z^{2}} \\
	s &= T(r) = G(z) \\
	T(r) &= \int_{0}^{r}p_r(w)dw \\
	&= A\int_{0}^{r}e^{-w}dw \\
	&= A(1-e^{-r}) \\
	G(z) &= \int_{0}^{z}p_z(t)dt \\
	&= B\int_{0}^{z}ze^{-t^{2}}dt \\
	&= \frac{B}{2}(1-e^{-z^{2}}) \\
	T(r) &= G(z) \\
	A(1-e^{-r}) &= \frac{B}{2}(1-e^{-z^{2}}) \\
	e^{-z^{2}} &= 1 - \frac{2A}{B}(1-e^{-r}) \\
	e^{z^{2}} &= \frac{1}{1 - \frac{2A}{B}(1-e^{-r})} \\
	z^{2} &= \ln\left(\frac{1}{1 - \frac{2A}{B}(1-e^{-r})}\right) \\
	z &= \sqrt{\ln\left(\frac{1}{1 - \frac{2A}{B}(1-e^{-r})}\right)}
	\end{align*}
    
    \pagebreak 

    \item \textbf{An image has the PDF shown below. It is desired to transform
    the intensity values of this image so that its PDF becomes $p_z(z)$.
    $p_z(z)$ is also shown below. Assume continuous quantities and find the
    transformation to perform this task.}
   
    \begin{align*}
	p_r(r) &= -2r + 2 \\
	p_z(z) &= 2z \\
	s &= T(r) = G(z) \\
	T(r) &= \int_{0}^{r}p_r(w)dw \\
	&= \int_{0}^{r}(-2w+2)dw \\
	&= -r^{2} + 2r \\
	G(z) &= \int_{0}^{z}p_z(t)dt \\
	&= \int_{0}^{z}2tdt \\
	&= z^{2} \\
	G(z) &= T(r) \\
	z^{2} &= -r^{2} + 2r \\
	z &= \sqrt{r(2-r)}
    \end{align*} 

    \item \textbf{Why do histogram equalization of a digital image usually not
    produce images with flat histograms?}

    With digital images, all we really do is create a look up table and change
    one intentsity value into another. If a bin within an image is very large,
    we would naturally want to break it up in order to create a flat histogram,
    but we cannot do this. How shall we decide which pixel of this value is
    shifted to another bin? We cannot know this, atleast without other image
    processing methods to aid in this endeavour. It is because this reality that
    histogram equalization does not usually result in a perfectly flat
    histogram.

    \item \textbf{A 50x70 image has 3-bit pixels. Its histogram $H(r)$ looks
    like a ramp as shown below. The counts in the histogram follow the formula
    $H(r)=kr$, where $k$ is a constant.}

    \begin{enumerate}[font=\bfseries, label=\alph*.]
    
	\item \textbf{Determine value of k.}
	
	\begin{align*}
	    M \times N &= \sum_{r=0}^{7}kr \\
	    k &= \frac{1}{M \times N}\cdot\sum_{r=0}^{7}r \\
	    &= \frac{28}{50 \times 70} \\
	    &= 125
	\end{align*}

	\item \textbf{Compute mean and standard deviation of this image from the
	histogram.}

	\begin{align*}
	    \bar{x} &= \frac{1}{M \times N}\cdot\sum_{r=0}^{7}rH(r) \\
	    &= \frac{1}{M \times N}\cdot\sum_{r=0}^{7}r^{2}k \\
	    &= \frac{k}{M \times N}\cdot\sum_{r=0}^{7}r^{2} \\
	    &= \frac{125}{50 \times 70}\cdot140 \\
	    &= 5
	\end{align*}

	\begin{align*}
	    s &= \sqrt{\frac{1}{N}\cdot\sum(x - \bar{x})^{2}} \\
	    &= \sqrt{\frac{1}{M \times N}\cdot\sum_{r=0}^{7}rk(r -
		\bar{x})^{2}}\\
	    &= \sqrt{\frac{125}{50 \times 70}\cdot\sum_{r=0}^{7}r(r -
		5)^{2}} \\
	    &= \sqrt{3} \\
	    &= 1.73
	\end{align*}
	
	\item \textbf{Compute the transformation function $s=T(r)$ that will
	equalize the histogram.}

	\begin{align*}
	    s_k = T(r_k) &= (L-1)\sum_{j=0}^{k}p_r(r_j) \\
	    &= \frac{(L-1)}{M \times N}\sum_{j=0}^{k}H(r_j) \\
	    &= \frac{(L-1)}{M \times N}\sum_{j=0}^{k}125j \\
	    &= \frac{(L-1)125}{M \times N}\sum_{j=0}^{k}j \\
	    &= \frac{(L-1)125}{M \times N}\cdot\frac{k(k+1)}{2} \\
	    &= \frac{7\cdot125}{50 \times 70}\cdot\frac{k(k+1)}{2} \\
	    &= 0.125\cdot k(k+1)
	\end{align*}
	
	*Note: In this case I used a property of summation to simplify the above
	relationship.

	\item \textbf{Compute the histogram $H(s)$ of the resulting image if it
	were transformed by $T$.}
	
	Tabulation of the histogram transformation:

	\begin{table}[H]
	    \centering
	    \begin{tabular}{|c|c|c|c|c|} \hline
	    $r_k$ & $n_k$ & $p_r(r_k)$ & $s_k$ & round($s_k$) \\ \hline
	    0 & 0   & 0.000 & 0.00 & 0 \\ \hline
	    1 & 125 & 0.036 & 0.25 & 0 \\ \hline
	    2 & 250 & 0.071 & 0.75 & 1 \\ \hline
	    3 & 375 & 0.107 & 1.50 & 2 \\ \hline
	    4 & 500 & 0.143 & 2.50 & 3 \\ \hline
	    5 & 625 & 0.179 & 3.75 & 4 \\ \hline
	    6 & 750 & 0.214 & 5.00 & 5 \\ \hline
	    7 & 875 & 0.250 & 7.00 & 7 \\ \hline
	      & 3500 & & & \\ \hline
	    \end{tabular}
	\end{table}
	
	Resulting histogram:

	\begin{table}[H]
	    \centering
	    \begin{tabular}{|c|c|} \hline
	    $s_k$ & $n_k$ \\ \hline
	    0 & 125  \\ \hline
	    1 & 250  \\ \hline
	    2 & 375  \\ \hline
	    3 & 500  \\ \hline
	    4 & 625  \\ \hline
	    5 & 750  \\ \hline
	    6 & 0    \\ \hline
	    7 & 875  \\ \hline
	      & 3500 \\ \hline
	    \end{tabular}
	\end{table}

	\item \textbf{Compute the mean and standard deviation of the resulting
	image.}
	
	The mean and standard deviation were calculated (on a spreadsheet) and
	were found to be 4.25 and 2.18 respectively.

	It is interesting to note that the mean of a perfectly flat histogram is
	3.5 with a standard deviation of . When the image histogram is
	equalized, it's mean and standard deviation approach these values.

    \end{enumerate}

\end{enumerate}
