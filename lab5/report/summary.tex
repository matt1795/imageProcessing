\section{Summary}

In this lab we explore sub-sampling in two dimensions. We see that concepts from
one dimension are not very different. For example, the closer the sample, the
greater distance there is in spectral images. 

There is a practical reason to keep samples far apart however, and that is to
reduce the memory taken up by a single image. As we shrink an image,
sub-sampling it, and then resize the image, there is an information loss, and
the large sample interval brings spectral images close together, causing
aliasing. Visualy this can look like very poor quality images, or distorted
images. In one of our excercizes we have an image that has a single frequency
and we sub-sample the image. By do so, and resizing the image for viewing, we
bring all those spectral images together, and the aliasing causes the image to
seem as if its rotating.
