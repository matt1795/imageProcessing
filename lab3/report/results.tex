\section{Results}

\subsection{Histograms}

\begin{figure}[H]
    \centering
    \includegraphics[scale=0.75]{histogram_compare.png}
    \caption{Histograms of Several Images}
\end{figure}

Barbara has a concentration of intensities in the middle, however has very few
at the minimum and maximum values. Tire spans the entire range of intensities,
but has a massive concentration at very dark levels. Pout has very concentrated
values around middle intensity levels, this will result in very low contrast.
Finally, eight has some values in the mid-range, but has a very large
concentration of intensity at high levels. Overall I expect tire to be the
darkest image and eight to be the brightest.

\subsection{Histogram Equalization}


\begin{figure}[H]
    \centering
    \includegraphics[scale=0.75]{barbara_compare.png}
    \caption{My histogram function vs. imhist of barbara}
\end{figure}

\begin{figure}[H]
    \centering
    \includegraphics[scale=0.75]{Tire_gray_compare.png}
    \caption{My histogram function vs. imhist of Tire\_gray}
\end{figure}

\begin{figure}[H]
    \centering
    \includegraphics[scale=0.75]{pout_gray_compare.png}
    \caption{My histogram function vs. imhist of pout\_gray}
\end{figure}

\begin{figure}[H]
    \centering
    \includegraphics[scale=0.75]{eight_gray_compare.png}
    \caption{My histogram function vs. imhist of eight\_gray}
\end{figure}

First I've tested my histogram function on our images. While the Matlab function
adds in some pretty plotting stuff, I have clearly scripted my function
correctly. In some cases it seems that Matlab cuts off view of some intensities
and in my plots you can see how many pixels there really are for a given
intensity.

\begin{figure}[H]
    \centering
    \includegraphics[scale=0.75]{pout_gray_equalize.png}
    \caption{Histogram Equalization of pout}
\end{figure}

\begin{figure}[H]
    \centering
    \includegraphics[scale=0.75]{Tire_gray_equalize.png}
    \caption{Histogram Equalization of Tire}
\end{figure}

\subsection{Histogram Matching}
