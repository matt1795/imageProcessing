\section{Summary}

Histogram processing allows us to use some very nonlinear operations to do some
useful things to images. We treat the histogram of images as the probability
density of random variables. Doing so allows us to use the math behind
transforming these density functions which is a powerful tool. 

First we started with simply generating the histogram of an image. We compared
how different images and their histograms in order to see the extra insight that
histograms give us.

Secondly, I developed a function to equalize image histograms. This is a very
general and powerful operation which maximizes the contrast within an image. It
can be used to easily improve old and faded images which were taken with old
tech cameras which did not provide a full range of contrast. It can also be used
to bring out detail within an image that is difficult to see.

Lastly, I build a histogram matching function which uses the two preceding
functions in this lab. I equalized an image, and calculated the transformation
function to equalize the output, and inverted that second equalization. Then I
mapped the equalized image to the histogram of the second image. This worked
successfully and I managed to match tire to barbara.

We only did our histogram processing on greyscale images. I am very interested
in looking into how histogram processing could be used in color images. It
initially looks like a good method to detect objects that may be the same color
using thresholding.
